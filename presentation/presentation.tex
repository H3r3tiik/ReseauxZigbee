\documentclass{beamer}

\usepackage[francais]{babel}
\usepackage[T1]{fontenc}
\usepackage[utf8]{inputenc}
\usepackage{beamerthemecs}
\usepackage{beamerouterthemecs}
\usepackage{beamerfontthemecs}
\usepackage{beamerinnerthemecs}
\usepackage{beamercolorthemecs}

\usetheme{cs}
\useoutertheme{cs}
\usefonttheme{cs}
\useinnertheme{cs}
\usecolortheme{cs}

\title{Les Réseaux Zigbee}
\author{\textbf{Thibault \textsc{Lengagne}, Sofian \textsc{Medbouhi} et Staninslas \textsc{Fechner}}}
\institute{Centrale Supélec - Campus de Rennes}

\begin{document}

  \begin{frame}
    \titlepage
  \end{frame}
  
  \AtBeginSection[] {
    \begin{frame}
      \frametitle{Plan}
      \tableofcontents[currentsection, hideothersubsections, pausesubsections]
    \end{frame} 
  }

  \section{Introduction}
  \begin{frame}
   \frametitle{Réseaux Zigbee}
   \textit{Windows Server} est la dénomination regroupant les systèmes d'exploitation orientés serveur de Microsoft :
   \begin{itemize}
    \item Windows NT
    \item Windows 2000 Server
    \item Windows Server 2003
    \item Windows Server 2008

   \end{itemize}
  \end{frame}
  
  \begin{frame}
   \frametitle{Fonctions principales}
   Depuis \textit{Windows 2000 Server} sont apparut les fonctionnalités suivantes :
   \begin{itemize}
    \item Active Directory, un annuaire d'organisation et de gestion des objets réseaux
    \item Un serveur web \textit{Internet Information Services} (IIS) intégré
    \item Terminal Server, un service permettant d'ouvrir une session à distance
    \item La prise en charge d'une grande quantité de RAM et de processeurs multiples
    \item La prise en charge de nombreux protocoles
    \item La gestion centralisée de clients multiples
    \item Une interface de gestion réseau
   \end{itemize}
  \end{frame}

  \begin{frame}
     \frametitle{Stratégie}
    \begin{itemize}
     \item Désigne la configuration logicielle du système par rapport aux utilisateurs
     \item Par défaut suite à une installation de \textit{Windows} : aucun stratégie n'est configurée
    \end{itemize}
    \begin{alertblock}{Conséquence dangereuse}
     Tout est permis en fonction des droits des groupes d'utilisateurs prédéfinis (administrateurs, utilisateurs, utilisateurs avec pouvoir, ...).
    \end{alertblock}
  \end{frame}
  
  \section{La norme 802.14.5}
  \begin{frame}
    \frametitle{Historique}
    \begin{itemize}
     \item \textit{Network Directory Services} (NTDS) à l'origine
     \item Présenté en 1996 pour la première fois
     \item Première utilisation en 1999 dans \textit{Windows 2000 Server}
     \item Amélioré constamment depuis
     \item Résulte d'une évolution de la BDD de comptes de domaine \textit{Security Account Manager} (SAM) et une mise en oeuvre de LDAP
     \item \textit{Active Directory} est donc un annuaire LDAP
    \end{itemize}
  \end{frame}
  
  \begin{frame}
    \frametitle{Objectifs}
    \begin{block}{Objectif principal}
     Fournir des services centralisés d'identification et d'authentification à un réseau d'ordinateurs utilisant le système Windows.
    \end{block}
    \begin{block}{Objectifs secondaires}
     \begin{itemize}
      \item Attribution et application de stratégies
      \item Distribution de logiciels
      \item Installation de mises à jour critiques par les administrateurs
     \end{itemize}
    \end{block}
  \end{frame}

  \section{Les couches 4 à 7}
  \begin{frame}
    \frametitle{Présentation}
    \begin{itemize}
     \item Permet de configurer des restrictions d'utilisation de \textit{Windows} 
     \item Permet de configurer des paramètres à appliquer soit sur un ordinateur donné, soit sur un compte utilisateur donné
     % plus précis concernant les comptes utilisateurs
    \end{itemize}
    D'où la possibilité d'agir sur :
    \begin{itemize}
     \item La définition d'un environnement
     \item Le déploiement de logiciels
     \item L'application des paramètres de sécurité
    \end{itemize}
  \end{frame}



  \section{Conclusion}
  
  \begin{frame}
    \begin{center}
      Merci de votre attention
    \end{center}
  \end{frame}

\end{document}