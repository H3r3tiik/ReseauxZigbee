\documentclass{beamer}

\usepackage[francais]{babel}
\usepackage[T1]{fontenc}
\usepackage[utf8]{inputenc}
\usepackage{beamerthemecs}
\usepackage{beamerouterthemecs}
\usepackage{beamerfontthemecs}
\usepackage{beamerinnerthemecs}
\usepackage{beamercolorthemecs}

\usetheme{cs}
\useoutertheme{cs}
\usefonttheme{cs}
\useinnertheme{cs}
\usecolortheme{cs}

\title{Sécurité dans les domaines MS Windows : Les GPO}
\author{\textbf{Thibault \textsc{Lengagne} et Valentin \textsc{Jaouen}}}
\institute{Centrale Supélec - Campus de Rennes}

\begin{document}

  \begin{frame}
    \titlepage
  \end{frame}
  
  \AtBeginSection[] {
    \begin{frame}
      \frametitle{Plan}
      \tableofcontents[currentsection, hideothersubsections, pausesubsections]
    \end{frame} 
  }

  \section{Introduction}
  \begin{frame}
   \frametitle{Windows Server}
   \textit{Windows Server} est la dénomination regroupant les systèmes d'exploitation orientés serveur de Microsoft :
   \begin{itemize}
    \item Windows NT
    \item Windows 2000 Server
    \item Windows Server 2003
    \item Windows Server 2008
    \item Windows Server 2008 R2
    \item Windows Server 2012 
    \item Windows Server 2012 R2
    \item Windows Server 2016
   \end{itemize}
  \end{frame}
  
  \begin{frame}
   \frametitle{Fonctions principales}
   Depuis \textit{Windows 2000 Server} sont apparut les fonctionnalités suivantes :
   \begin{itemize}
    \item Active Directory, un annuaire d'organisation et de gestion des objets réseaux
    \item Un serveur web \textit{Internet Information Services} (IIS) intégré
    \item Terminal Server, un service permettant d'ouvrir une session à distance
    \item La prise en charge d'une grande quantité de RAM et de processeurs multiples
    \item La prise en charge de nombreux protocoles
    \item La gestion centralisée de clients multiples
    \item Une interface de gestion réseau
   \end{itemize}
  \end{frame}

  \begin{frame}
     \frametitle{Stratégie}
    \begin{itemize}
     \item Désigne la configuration logicielle du système par rapport aux utilisateurs
     \item Par défaut suite à une installation de \textit{Windows} : aucun stratégie n'est configurée
    \end{itemize}
    \begin{alertblock}{Conséquence dangereuse}
     Tout est permis en fonction des droits des groupes d'utilisateurs prédéfinis (administrateurs, utilisateurs, utilisateurs avec pouvoir, ...).
    \end{alertblock}
  \end{frame}
  
  \section{Active Directory}
  \begin{frame}
    \frametitle{Historique}
    \begin{itemize}
     \item \textit{Network Directory Services} (NTDS) à l'origine
     \item Présenté en 1996 pour la première fois
     \item Première utilisation en 1999 dans \textit{Windows 2000 Server}
     \item Amélioré constamment depuis
     \item Résulte d'une évolution de la BDD de comptes de domaine \textit{Security Account Manager} (SAM) et une mise en oeuvre de LDAP
     \item \textit{Active Directory} est donc un annuaire LDAP
    \end{itemize}
  \end{frame}
  
  \begin{frame}
    \frametitle{Objectifs}
    \begin{block}{Objectif principal}
     Fournir des services centralisés d'identification et d'authentification à un réseau d'ordinateurs utilisant le système Windows.
    \end{block}
    \begin{block}{Objectifs secondaires}
     \begin{itemize}
      \item Attribution et application de stratégies
      \item Distribution de logiciels
      \item Installation de mises à jour critiques par les administrateurs
     \end{itemize}
    \end{block}
  \end{frame}
  
  \begin{frame}
    \frametitle{Structure Active Directory}
    \begin{block}{Active Directory}
      Service d'annuaire utilisé pour stocker des informations relatives aux ressources réseau sur un domaine.
    \end{block}    
    \begin{block}{Structure Active Directory}
      Organisation hiérarchisée d'objets classés en trois catégories :
      \begin{itemize}
       \item les ressources : imprimantes, ordinateurs ...
       \item les services : courrier électronique, intranet, ...
       \item les utilisateurs : comptes utilisateurs et groupes
      \end{itemize}
      Ces objets peuvent être regroupés en :
      \begin{itemize}
       \item unités d'organisation permettant l'application des GPO
       \item groupes universels, globaux ou locaux
       \item arbres et forêts
      \end{itemize}
    \end{block}
  \end{frame}

  \section{Stratégie de groupe}
  \begin{frame}
    \frametitle{Présentation}
    \begin{itemize}
     \item Permet de configurer des restrictions d'utilisation de \textit{Windows} 
     \item Permet de configurer des paramètres à appliquer soit sur un ordinateur donné, soit sur un compte utilisateur donné
     % plus précis concernant les comptes utilisateurs
    \end{itemize}
    D'où la possibilité d'agir sur :
    \begin{itemize}
     \item La définition d'un environnement
     \item Le déploiement de logiciels
     \item L'application des paramètres de sécurité
    \end{itemize}
  \end{frame}
  
  \begin{frame}
   \frametitle{Exemple de stratégie de groupe}
   \begin{itemize}
    \item Menu démarrer et Barre des tâches
    \begin{enumerate}
     \item Suppression du menu Documents dans le menu Démarrer
     \item Suppression des Connexions réseau et accès distant du menu Démarrer
     \item Suppression du menu Exécuter dans le menu Démarrer
     \item Désactivation de la fermeture de session dans le menu Démarrer
     \item Désactivation de la commande Arrêter
    \end{enumerate}
    \item Panneau de configuration
    \begin{enumerate}
     \item Désactivation du Panneau de configuration 
     \item Masque de certaines applications du Panneau de configuration
    \end{enumerate}
    \item Système
    \begin{enumerate}
     \item Activation des quotas de disque
     \item Désactivation des outils de modifications du Registre
     \item Désactivation de l'invite de commandes
    \end{enumerate}
    \item Internet Explorer
    \begin{enumerate}
     \item Désactivation de la modification des paramètres de la page de démarrage
    \end{enumerate}
   \end{itemize}
  \end{frame}  
  
  \section{Mise en oeuvre des stratégies de groupe : les GPOs}
  
  \begin{frame}
   \frametitle{Présentation}
   \begin{itemize}
    \item GPO signifie \textit{Group Policy Object}
    \item Apparus avec \textit{Windows 2000 Server} avec l'\textit{Active Directory}
    \item Un GPO est un objet \textit{Active Directory}
    \item Il contient un ensemble de paramètres applicables à un utilisateur ou à un ordinateur
    \item Les GPOs sont accessibles dans tous le domaine, dans le dossier SYSVOL
   \end{itemize}
   \begin{block}{SYSVOL ?}
    Dossier partagé sur un volume NTFS entre les contrôleurs de domaine et avec les clients du domaine.
   \end{block}
  \end{frame}

  \begin{frame}
   \frametitle{Avantages des GPO}
   \begin{itemize}
    \item Configurent de manière automatisée et centralisée les postes de travail et les serveurs \textit{Windows} d'un environnement donné
    \item La configuration des postes n'est pa stockée localement. Il suffit d'ajouter un poste un GPO existant et la configuration s'applique
    \item La suppression d'une GPO restaure les paramètres locaux
    \item Il est presque possible de configurer n'importe quel paramètre par GPO
    \item Configuration des postes uniforme
    \item Les GPO sont réappliquées à intervalle régulier
   \end{itemize}
  \end{frame}
  
  \begin{frame}
   \frametitle{Limites des GPO}
   \begin{itemize}
    \item La gestion des GPOs n'est pas chose aisée
    \begin{itemize}
      \item Il faut être sûr d'appliquer le ou les bons GPOs aux bons ordinateurs et/ou utilisateurs
      \item Beaucoup d'administrateurs sont confrontés à ce problème lorsqu'ils mettent en oeuvre les GPOs
    \end{itemize}
    \item Si chaque poste ou chaque utilisateur du réseau nécessite une configuration particulière, les GPOs ne sont plus vraiment efficaces
   \end{itemize}
  \end{frame}


  \section{Conclusion}
  
  \begin{frame}
    \begin{center}
      Merci de votre attention
    \end{center}
  \end{frame}

\end{document}