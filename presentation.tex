\documentclass{beamer}

\usepackage[francais]{babel}
\usepackage[T1]{fontenc}
\usepackage[utf8]{inputenc}
\usepackage{beamerthemecs}
\usepackage{beamerouterthemecs}
\usepackage{beamerfontthemecs}
\usepackage{beamerinnerthemecs}
\usepackage{beamercolorthemecs}

\usetheme{cs}
\useoutertheme{cs}
\usefonttheme{cs}
\useinnertheme{cs}
\usecolortheme{cs}

\title{Sécurité dans les domaines MS Windows : Les GPO}
\author{\textbf{Thibault \textsc{Lengagne} et Valentin \textsc{Jaouen}}}
\institute{Centrale Supélec - Campus de Rennes}

\begin{document}

  \begin{frame}
    \titlepage
  \end{frame}
  
  \AtBeginSection[] {
    \begin{frame}
      \frametitle{Plan}
      \tableofcontents[currentsection, hideothersubsections, pausesubsections]
    \end{frame} 
  }

  \section{Introduction}
  \begin{frame}
    \begin{itemize}
     \item 
    \end{itemize}
  \end{frame}
  
  \section{Active Directory}
  \begin{frame}
    \frametitle{Historique}
    \begin{itemize}
     \item \textit{Network Directory Services} (NTDS) à l'origine
     \item Présenté en 1996 pour la première fois
     \item Première utilisation en 1999 dans \textit{Windows 2000 Server Edition}
     \item Amélioré constamment depuis
     \item Résulte d'une évolution de la BDD de comptes de domaine \textit{Security Account Manager} (SAM) et une mise en oeuvre de LDAP
     \item \textit{Active Directory} est donc un annuaire LDAP
    \end{itemize}
  \end{frame}
  
  \begin{frame}
    \frametitle{Objectifs}
    \begin{block}{Objectif principal}
     Fournir des services centralisés d'identification et d'authentification à un réseau d'ordinateurs utilisant le système Windows.
    \end{block}
    \begin{block}{Objectifs secondaires}
     \begin{itemize}
      \item Attribution et application de stratégies
      \item Distribution de logiciels
      \item Installation de mises à jour critiques par les administrateurs
     \end{itemize}
    \end{block}
  \end{frame}
  
  \begin{frame}
    \frametitle{Structure Active Directory}
    \begin{block}{Active Directory}
      Service d'annuaire utilisé pour stocker des informations relatives aux ressources réseau sur un domaine.
    \end{block}    
    \begin{block}{Structure Active Directory}
      Organisation hiérarchisée d'objets classé en trois catégories :
      \begin{itemize}
       \item les ressources : imprimantes, routeur, ...
       \item les services : courrier électronique, intranet, ...
       \item les utilisateurs : comptes utilisateurs et groupes
      \end{itemize}
    \end{block}
  \end{frame}

  \section{Stratégie de groupe}
  
  
  \section{Conclusion}
  
  \begin{frame}
    \begin{center}
      Merci de votre attention
    \end{center}
  \end{frame}

\end{document}